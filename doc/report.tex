\documentclass{article}


\usepackage[backend=bibtex, style=ieee]{biblatex}
\usepackage{graphicx}
\usepackage{hyperref}
\usepackage{siunitx}
\usepackage[table]{xcolor}


\bibliography{report}
\title{Human identification via gait in unconstrained videos}
\author{Miroslav Vitkov \\ sir.vorac@gmail.com}


\begin{document}
\maketitle


\begin{abstract}
Gait is a prominent biometric with similar importance to face recognition.
The latter is feasable in the presence of heavy noise and can utilize inexpensive equipment.
This report explores the feasability of the former when subjected to the same constraints.
\end{abstract}


\section{Introduction}
The interdisciplinary field of human identification via gait is now 42 years old.
Initially, glowing markers were used\cite{begin} for 38\% recognition accuracy, compared to random chance of 16.7\%.

Further development split into model-free and model-based methods.
Model-free methods look at pixels, one notable work being the gait energy model\cite{energy}.
Model-based approaches use a strong prior, e.g. a pendulum model of legs\cite{pendulum}.

Modern methods are dominated by motion vector estimated approaches, such as \cite{pyramid}
A recent improvement to \cite{pyramid} demonstrates human recognition accuracy of 96\% using a fusion of a visible spectrum camera, a Microsoft Kinetic and a wearable accelerometer\cite{robust}.


\section{Goal}
Consider a random video of people walking - a movie, a running competition, a protesting crowd.
The purpose of this work is to extract human stick-figure model of humans from the video.
Not only the subject is moving but also the background and the verycamera.


\section{Method}
\begin{enumerate}
\item{Process one frame at a time.}
\item{Detect prominent features - intersecting edges of high contrast .}
\item{Compute scene translation and rotation relative to the camera (without loss of generality, assume a static camera). Exclude outliers, such as the walking person.}
\item{Parallelly to all the above, run a HAAR standing person detector on the frame.}
\item{Now consider all frames of the video together. Interpolate and denoise the human figure movement.}
\end{enumerate}


\section{Dataset}
kur\cite{largest}


\printbibliography


\end{document}

