\documentclass{article}


\usepackage[backend=bibtex, style=ieee]{biblatex}
\usepackage{graphicx}
\usepackage{hyperref}
\usepackage{siunitx}
\usepackage[table]{xcolor}


\bibliography{report}
\title{Human silhouette segmentation in unconstrained videos}
\author{Miroslav Vitkov \\ sir.vorac@gmail.com}


\begin{document}
\maketitle


\begin{abstract}
Gait is a prominent biometric with similar importance to face recognition.
It is feasable in the presence of heavy noise like darkness or low quality cameras.
The input to the recogniser algorithm is usually a segmented and deshadowed human silhouette.
This report presents an implementation of the segmentation step.
We use Histogram of Oriented Gradients for dimensionality reduction  and a Support Vector Machine for pixel classification.
\end{abstract}


\section{Introduction}
\subsection{Gait}
The interdisciplinary field of human identification via gait is now 42 years old.
Initially, glowing markers were used\cite{begin} for 38\% recognition accuracy, compared to random chance of 16.7\%.

Further development split into model-free and model-based methods.
Model-free methods look at pixels, one notable work being the gait energy model\cite{energy}.
Model-based approaches use a strong prior, e.g. a pendulum model of legs\cite{pendulum}.

Modern methods are dominated by motion vector estimated approaches, such as \cite{pyramid}
A recent improvement to \cite{pyramid} demonstrates human recognition accuracy of 96\% using a fusion of a visible spectrum camera, a Microsoft Kinetic and a wearable accelerometer\cite{robust}.


\subsection{Silhouette}
One common preprocessing step is adaptively tresholed background subtraction \cite{vehicle}.
Piels in consecutive frames are compared and if the difference is less that a treshold value, classified as background and thus not interesting.
The treshold is generated by an exponential filter.
The difference is computed by sum of absolute difference (SAD) \cite{background}.
This allows the procedure to ignore periodic noise e.g waving tree leaves and low-frequency noise e.g. parked car or change of illumination level.

The Histograms of Oriented Gradients (HOG)\cite{hog} descriptor encodes an an image's local colour gradients.
It works in a sliding window or square or circular shape.
Horizontal and vertical derivatives are calculated with a centered mask (1 0 1).
All generated values in the window are summarized into one tupple (direction, magnitude).
This approachis believed to work better than the earlier Local Binary Patterns.


\section{Dataset}
Numerous human tracking outdoor datasets are described in the literature\cite{datasets0}\cite{datasets1}.
Notable is the occlusion dataset\cite{datasets2}, providing experience close to real world outdoor survailence.
The AVA indoor dataset\cite{ava} consists of a large number (1200) single view videos.
Because of that it was chosen for this project.



\section{Method}
\begin{enumerate}
\item{Process one frame at a time.}
\item{Detect prominent features - intersecting edges of high contrast .}
\item{Compute scene translation and rotation relative to the camera (without loss of generality, assume a static camera). Exclude outliers, such as the walking person.}
\item{Parallelly to all the above, run a HAAR standing person detector on the frame.}
\item{Now consider all frames of the video together. Interpolate and denoise the human figure movement.}
\end{enumerate}


\printbibliography


\end{document}

